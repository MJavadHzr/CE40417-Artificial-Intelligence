\documentclass[a4paper, 12pt]{article}
\usepackage{temp}
\usepackage{epsfig,graphicx,subfigure,amsthm,amsmath, float, xcolor, changepage, mathtools, textcomp, hyperref, bm, amssymb, tcolorbox, tikz, setspace}
\usepackage{array}
\usepackage[shortlabels]{enumitem}
\usepackage[stable]{footmisc}
\usepackage{xepersian}
\settextfont[Scale=1]{XBZar}
%\setdigitfont{XBZar}
\setlatintextfont[Scale=0.9]{Times New Roman}
\hypersetup{
	colorlinks=true,
	urlcolor=blue!70!black
}

\newcolumntype{?}{!{\vrule width 1pt}}

\doublespacing
\begin{document}
	\handout
	{هوش مصنوعی}
	{نیم‌سال اول ۰۱\lr{-}۰۰}
	{دکتر محمدحسین رهبان}
	{دانشکده مهندسی کامپیوتر}
	{تمرین چهارم - بخش دوم}
	{محمدجواد هزاره}
	{98101074}
	\noindent
	\\[-6em]
	\section*{سوال ۱}
	\begin{enumerate}[A)]
		\item
		\begin{itemize}
			\item
			این رابطه لزوما برقرار نیست. اگر مسیر
			$P-W-S-R-K$
			را در نظر بگیریم، تمام سه‌تایی‌های این مسیر فعال هستند. بنابراین نمی‌توان از روی ساختار گراف به استقلال متغیرهای $P$ و $K$ به شرط $W$ پی برد.
			\item
			این رابطه نیز لزوما برقرار نیست. مسیر
			$P-W-K-R-S$
			مسیری فعال است چرا که تمام سه‌تایی‌های این مسیر فعال هستند. بنابراین در این مورد نیز نمی‌توان از روی ساختار گراف به استقلال شرطی پی برد.
			\item
			مسیر
			$P-H-M-C$
			مسیری فعال از $P$ به $C$ است و در نتیجه در این مورد هم مشابه موارد قبل ممکن است متغیرهای $P$ و $C$ وابسته‌ی شرطی باشند.
		\end{itemize}
		\item
		\begin{itemize}
			\item
			\item
			\item
		\end{itemize}
	\end{enumerate}
\end{document}



