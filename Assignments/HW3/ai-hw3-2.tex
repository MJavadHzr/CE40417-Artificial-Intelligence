\documentclass[a4paper, 12pt]{article}
\usepackage{temp}
\usepackage{epsfig,graphicx,subfigure,amsthm,amsmath, float, xcolor, changepage, mathtools, textcomp, hyperref, bm, amssymb, tcolorbox, tikz, setspace}
\usepackage{array}
\usepackage[shortlabels]{enumitem}
\usepackage[stable]{footmisc}
\usepackage{xepersian}
\settextfont[Scale=1]{XBZar}
%\setdigitfont{XBZar}
\setlatintextfont[Scale=0.9]{Times New Roman}
\hypersetup{
	colorlinks=true,
	urlcolor=blue!70!black
}

\newcolumntype{?}{!{\vrule width 1pt}}

\doublespacing
\begin{document}
\handout
{هوش مصنوعی}
{نیم‌سال اول ۰۱\lr{-}۰۰}
{دکتر محمدحسین رهبان}
{دانشکده مهندسی کامپیوتر}
{تمرین سوم - بخش دوم}
{محمدجواد هزاره}
{98101074}
\noindent
\\[-6em]
\section*{سوال ۱}
\begin{enumerate}[آ)]
	\item
	اگر به جای راس \lr{E} مقدار امید ریاضی آن که برابر 
	\[
	\E[E] = \frac{1}{2} 120 + \frac{1}{2}(-100) = 10
	\]
	است را قرار داده و سپس الگوریتم \lr{minmax} را اجرا کنیم، برای راس ریشه به عدد $10$ می‌رسیم. بنابراین امید ریاضی پولی که مت بدست می‌آورد و پت از دست می‌دهد برابر $10$ خواهد بود.
	\item
	از آن‌جایی که پت اطلاعاتی در رابطه با راس \lr{E} ندارد، از امید ریاضی آن استفاده خواهد کرد و در نتیجه راس \lr{E} را انتخاب خواهد کرد. از طرفی اگر مت بداند در راس \lr{E} مقدار $120$ قرار داشته، او نیز همین راس را انتخاب خواهد کرد و به مقدار $120$ می‌رسد. و اگر بداند که این راس مقدار $-100$ داشته، راس $-20$ را انتخاب خواهد کرد و به مقدار $-20$ خواهد رسید.
	\item
	با توجه به این که راس \lr{E} به احتمال $0.5$ مقدار $120$ و به احتمال $0.5$ مقدار $-100$ را دارد، مت با دانستن مقدار این راس به طور میانگین پولی برابر با
	\[
	\frac{1}{2} 120 + \frac{1}{2}(-20) = 50
	\]
	بدست خواهد آورد. بنابراین به طور میانگین، اطلاع داشتن از راس \lr{E} به مقدار $40$ واحد برای مت مفید است.
	\item
	در صورتی که هر دو از گره \lr{E} آگاه باشند، اگر این راس مقدار $120$ را داشته باشد، ریشه مقدار $60$ را به خود خواهد گرفت و اگر مقدار $-100$ داشته باشد، ریشه مقدار $-20$ می‌گیرد.
	\item
	با توجه به قسمت قبل، اگر هر دو از گره \lr{E} اطلاع داشته باشند، به طور میانگین پولی که مت برنده می‌شود برابر
	\[
	\frac{1}{2} 60 + \frac{1}{2} (-20) = 20
	\]
	خواهد بود. بنابراین در مقایسه با حالتی که هیچ یک از این گره اطلاع ندارند (قسمت آ)، آگاهی هر دو نسبت به این گره به نفع مت، و ناآگاه بودن هر دو نسبت به این راس به نفع پت خواهد بود.
\end{enumerate}
\end{document}



