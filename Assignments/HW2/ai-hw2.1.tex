\documentclass[a4paper, 11pt]{article}
\usepackage{temp}
\usepackage{epsfig,graphicx,subfigure,amsthm,amsmath, float, xcolor, changepage, mathtools, textcomp, hyperref, bm, amssymb, tcolorbox, tikz, setspace}
\usepackage[shortlabels]{enumitem}
\usepackage[stable]{footmisc}
\usepackage{xepersian}
\settextfont[Scale=1.2]{XBZar}
%\setdigitfont{XBZar}
\setlatintextfont[Scale=1.1]{Times New Roman}
\hypersetup{
	colorlinks=true,
	urlcolor=blue!70!black
}

\doublespacing
\begin{document}
\handout
{هوش مصنوعی}
{نیم‌سال اول ۰۱\lr{-}۰۰}
{دکتر محمدحسین رهبان}
{دانشکده مهندسی کامپیوتر}
{تمرین دوم - بخش اول}
{محمدجواد هزاره}
{98101074}
\noindent
\\ [-5em]
\section*{سوال ۱}
اگر تابع هیوریستیکی به صورت 
\[
h(s) = d(s)_{M} + c
\]
که $d(s)_M$ جمع فاصله منهتنی کاشی‌ها تا مکان درست آن‌ها و $c$ ثابت مثبتی است، آن‌گاه خواسته‌ی مسئله برقرار می‌شود. این هیوریستیک در بعضی مواقع هزینه‌ی بیش‌تری را پیش‌بینی می‌کند ولی در نهایت به هدفی می‌رسد که نهایتا $c$ واحد با هدف بهینه فاصله خواهد داشت.

به طور کلی می‌توان اثبات کرد که اگر تابع هیوریستیکی داشته باشیم که نهایتا $c$ واحد از $h^*$ بیش‌تر پیش‌بینی کند، به هدفی می‌رسیم که نهایتا $c$ واحد با هدف بهینه فاصله خواهد داشت.

\begin{proof}
	برای اثبات از برهان خلف استفاده می‌کنیم؛ فرض کنیم $A^*$ را با استفاده از هیوریستیک $\bar{h}$ اجرا کرده‌ایم که به ازای هر حالت $s$ برای آن داریم
	$\bar{h}(s) \le h^*(s) + c$.
	حال فرض کنیم جواب اجرای این الگوریتم خواسته‌ی ما را برآورده نمی‌کند، یا به عبارتی هدفی که این الگوریتم به آن می‌رسد بیش‌تر از $c$ واحد با هدف بهینه فاصله دارد. اگر این الگوریتم هدف $G_1$ را باز کرده باشد، داریم
	$g(G_1) > g^* + c$
	که $g^*$ هزینه‌ی رسیدن به هدف بهینه است. بنابراین برای
	$f(G_1)$
	داریم:
	\[
	\begin{aligned}
		f(G_1)	&= g(G_1) + \bar{h}(G_1) \\
				&\ge g(G_1) \\
				&> g^* + c
	\end{aligned}
	\]
	 فرض کنیم هدف بهینه $G_2$ بوده که طبیعتا باز نشده است و هم‌چنین به \lr{fringe} نیز اضافه نشده است. بنابراین در مسیر ریشه به $G_2$، آخرین راسی که باز نشده و به \lr{fringe} اضافه شده است را در نظر می‌گیریم و این راس را با $n$ نشان می‌دهیم. برای این راس داریم:
	 \[
	 \begin{aligned}
	 	f(n)	&= g(n) + \bar{h}(n) \\
	 			&= \big(g^* - h^*(n)\big) + \bar{h}(n) \\
	 			&\le g^* + c &\qquad (\bar{h}(s) \le h^*(s) + c)
	 \end{aligned}
	 \]
	 که در خط دوم از این حقیقت استفاده شده که $n$ در مسیر ریشه به $G_2$ قرار دارد. بنابراین با توجه به $f(G_1)$ خواهیم داشت که 
	 $f(n) < f(G_1)$
	 و این با باز شدن $G_1$ در تناقض است چرا که نخست راسی باز می‌شود که $f$ کم‌تری داشته باشد. بنابراین فرض خلف باطل و حکم اثبات می‌شود.
\end{proof}



\end{document}



