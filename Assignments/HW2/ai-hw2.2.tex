\documentclass[a4paper, 12pt]{article}
\usepackage{temp}
\usepackage{epsfig,graphicx,subfigure,amsthm,amsmath, float, xcolor, changepage, mathtools, textcomp, hyperref, bm, amssymb, tcolorbox, tikz, setspace}
\usepackage[shortlabels]{enumitem}
\usepackage[stable]{footmisc}
\usepackage{xepersian}
\settextfont[Scale=1]{XBZar}
%\setdigitfont{XBZar}
\setlatintextfont[Scale=0.9]{Times New Roman}
\hypersetup{
	colorlinks=true,
	urlcolor=blue!70!black
}

\doublespacing
\begin{document}
\handout
{هوش مصنوعی}
{نیم‌سال اول ۰۱\lr{-}۰۰}
{دکتر محمدحسین رهبان}
{دانشکده مهندسی کامپیوتر}
{تمرین دوم - بخش دوم}
{محمدجواد هزاره}
{98101074}
\noindent
\\ [-5em]
\section*{سوال ۱}
در این روش متغیر $t$ را از یک شروع کرده و یکی یکی زیاد می‌کنیم و تا زمانی که دما یا همان متغیر $T$ به صفر برسد الگوریتم را ادامه می‌دهیم. بنابراین مراحل اجرای الگوریتم به صورت زیر خواهد بود:
\begin{enumerate}[1.]
	\item
	در این مرحله $t = 1$ و برای وضعیت کنونی داریم
	$value(S_{current}) = -1$.
	برای $T$ نیز داریم $T = 2$. برای انتخاب تصادفی یکی از همسایه‌های وضعیت کنونی نیز با توجه به صورت سوال خواهیم داشت
	$value(S_{new}) = -2$.
	بنابراین 
	$\Delta E = -2 + 1 = -1$
	و از آنجایی که اختلاف ارزش وضعیت‌ها منفی‌ست، به صورت احتمالاتی این وضعیت را انتخاب خواهیم کرد. احتمال انتخاب شدن وضعیت جدید برابر 
	$e^{\frac{\Delta E}{T}} = e^{-\frac{1}{2}} \approx 0.6$
	خواهد بود که چون بیش‌تر از $0.5$ است این وضعیت انتخاب خواهد شد. بنابراین استیت جدید ما وضعیتی خواهد بود که ارزش آن $-2$ است. 
	\item
	در این مرحله $t = 2$ و با توجه به زمان‌بند داده شده داریم $T = 1$. با توجه به مرحله قبل، برای ارزش وضعیت کنونی داریم
	$value(S_{current} = -2)$.
	به طور مشابه برای وضغیت جدید انتخابی نیز داریم
	$value(S_{new} = -2)$.
	پس اختلاف ارزش‌ها برابر
	$\Delta E = 0$
	خواهد بود. در این حالت چون
	$\Delta E > 0$
	نیست، انتخاب به صورت احتمالاتی صورت می‌گیرد. احتمال انتخاب شدن وضعیت جدید برابر 
	$e^{\frac{0}{T}} = 1$
	خواهد بود و درنتیجه وضعیت جدید انتخاب می‌شود. بنابراین در پایان این مرحله برای وضعیت کنونی داریم 
	$value(S_{current} = -2)$.
	\item
	در این مرحله $t=3$ و در نتیجه با توجه به زمان‌بند داده شده $T=0$ خواهد بود. در نتیجه در این مرحله الگوریتم به پایان می‌رسد و وضعیتی با ارزش $-2$ خروجی داده می‌شود.
\end{enumerate}
\section*{سوال ۲}
\begin{enumerate}[آ)]
	\item
	برای این منظور باید تابع داده شده محدب باشد. برای بررسی محدب بودن تابع باید ماتریس \lr{Hessian} آن را بررسی کنیم. برای این ماتریس داریم:
	\[
	H_f = \left(\begin{array}{cc}
		\dfrac{\partial^2 f}{\partial x_1^2} & \dfrac{\partial^2 f}{\partial x_1\partial x_2} \\[0.75em]
		\dfrac{\partial^2 f}{\partial x_2\partial x_1} & \dfrac{\partial^2 f}{\partial x_2^2}
	\end{array}\right) = \left(\begin{array}{cc}
		12x_1^2 & 0 \\
		0 & 4
	\end{array}\right)
	\]
	این ماتریس در تمام نقاط
	$(x_1, x_2)$
	مثبت نیمه معین است، چرا که دترمینان‌های مربعی آن در این نقاط نامنفی است. بنابراین تابع محدب بوده و با اجرای الگوریتم
	\lr{gradient descent}
	و با انتخاب مناسب $\alpha$ به نقطه کمینه سراسری همگرا می‌شویم.
	\item
	برای گرادیان $f$ داریم:
	\[
	\nabla f = (4x_1 ^3 + 1, 4x_2)
	\]
	پس مراحل اجرای الگوریتم با
	$\alpha = 0.0001$
	به صورت زیر خواهد بود:
	\begin{enumerate}[1.]
		\item
		مقدار گرادیان در نقطه
		$x_0 = (-1,0)$
		برابر 
		$(-3,0)$
		خواهد بود، بنابراین برای مقدار جدید $x$ داریم:
		\[
		x_{1} = x_{0} - \alpha \nabla f = (-1,0) - 0.0001(-3,0) = (-0.9997,0) 
		\]
		\item
		در این مرحله گرادیان در نقطه $x_1$ تقریبا برابر 
		$(-2.9964,0)$
		خواهد بود، پس برای $x$ جدید داریم:
		\[
		x_2 = x_1 - \alpha \nabla f = (-0.9997,0) - 0.0001(-2.9964,0) = (-0.9994,0)
		\]
		\item
		در این مرحله نیز گرادیان در نقطه $x_2$ تقریبا برابر
		$(-2.9928,0)$
		خواهد بود که درنتیجه برای $x_3$ داریم:
		\[
		x_3 = x_2 - \alpha \nabla f = (-0.9994,0) - 0.0001(-2.9928,0) = (-0.9991,0)
		\]
	\end{enumerate}
	با توجه به گرادیان $f$ می‌دانیم این تابع در نقطه‌ای با مختصات تقریبی 
	$(-0.63,0)$
	مقدار کمینه را به خود می‌گیرد. پس همانطور که دیده می‌شود اجرای الگوریتم با 
	$\alpha = 0.0001$
	بسیار کند به سمت نقطه کمینه حرکت می‌کند.
	\item
	مراحل اجرای الگوریتم با
	$\alpha = 1$
	به صورت زیر خواهد بود:
	\begin{enumerate}[1.]
		\item
		در نقطه شروع گرادیان
		$(-3,0)$
		است، پس داریم:
		\[
		x_1 = x_0 - \alpha \nabla f = (-1,0) - (-3,0) = (2,0)
		\]
		\item
		در نقطه $x_1$ گرادیان برابر
		$(33,0)$
		خواهد بود، پس:
		\[
		x_2 = x_1 - \alpha \nabla f = (2,0) - (33,0) = (-31,0)
		\]
		\item
		در $x_2$ نیز گرادیان برابر
		$(-119163,0)$
		خواهد بود :) که در نتیجه داریم:
		\[
		x_3 = x_2 - \alpha \nabla f = (-31,0) - (-119163,0) = (119132,0)
		\]
	\end{enumerate}
	همانطور که دیده می‌شود با 
	$\alpha = 1$
	در هر مرحله بیش‌تر از نقطه کمینه فاصله می‌گیریم و درنتیجه با انتخاب این مقدار برای 
	$\alpha$
	الگوریتم به مقدار کمینه \underline{همگرا نخواهد شد}.
	\item
	با توجه به قسمت (ب) و (پ) نتیجه می‌گیریم که اگر $\alpha$ را خیلی کوچک انتخاب کنیم با سرعتی بسیار کند به نقطه کمینه همگرا خواهیم شد و اگر $\alpha$ را خیلی بزرگ انتخاب کنیم به طور کلی همگرایی به نقطه کمینه را از دست خواهیم داد. بنابراین انتخاب $\alpha$ چیزی میان این دو مقدار و حدود $0.1$ مقدار مناسبی خواهد بود.
\end{enumerate}
\end{document}



