\documentclass[a4paper, 11pt]{article}
\usepackage{AItemp}
\usepackage{epsfig,graphicx,subfigure,amsthm,amsmath, float, xcolor, changepage, mathtools, textcomp, hyperref, bm, amssymb, tcolorbox, tikz, setspace}
\usepackage[shortlabels]{enumitem}
\usepackage[stable]{footmisc}
\usepackage{xepersian}
\settextfont[Scale=1.2]{XBZar}
%\setdigitfont{XBZar}
\setlatintextfont[Scale=1.1]{Times New Roman}
\hypersetup{
	colorlinks=true,
	urlcolor=blue!70!black
}

\doublespacing
\begin{document}
\handout
{هوش مصنوعی}
{نیم‌سال اول ۰۱\lr{-}۰۰}
{دکتر محمدحسین رهبان}
{دانشکده مهندسی کامپیوتر}
{تمرین اول - بخش اول}
{محمد جواد هزاره}
{98101074}
\noindent
\\ [-5em]
\section*{سوال ۱}

\begin{enumerate}[(آ)]
	\item
	\begin{enumerate}[1.]
		\item \textbf{راننده تاکسی خودکار}
		\begin{itemize}
			\item \lr{Performance Measure}:
			رسیدن به مقصد / زمان رسیدن به مقصد / هزینه رسیدن به مقصد / استحلاک ماشین و مسافت طی شده / رعایت قوانین و مقررات رانندگی
			\item \lr{Environment}:
			جاده‌ها / عابران / سایر خودرو‌ها
			\item \lr{Actuator}:
			فرمان / گاز / بوق / چراغ‌ها / صفحه نمایش
			\item \lr{Sensors}:
			دوربین / جی‌پی‌اس / میکروفن / سرعت‌سنج / رابطی برای دریافت اطلاعات مقصد مسافر
		\end{itemize}

		\item \textbf{ربات انجام‌دهنده بازی دوز به صورت فیزیکی}
		\begin{itemize}
			\item \lr{Performance Measure}:
			ردیف کردن مهره‌های خودی / جلوگیری از ردیف شدن مهره‌های حریف
			\item \lr{Environment}:
			صفحه بازی / محل قرار گرفتن مهره‌ها
			\item \lr{Actuator}:
			بازوی الکتریکی
			\item \lr{Sensors}:
			دوربین
		\end{itemize}
	
	\end{enumerate}
	\item
	\begin{enumerate}[1.]
		\item \textbf{ربات انجام‌دهنده بازی دوز به صورت نرم‌افزاری}\\
		\lr{Fully Observable / Strategic / Sequential / Static / Multi Agent / Discrete}
		\item \textbf{شطرنج زمان‌دار}\\
		\lr{Fully Observable / Strategic / Sequential / Semi-Dynamic / Multi Agent / Discrete}
	\end{enumerate}
\end{enumerate}
\section{سوال ۲}
\begin{itemize}
	\item حالات:
	هر چیدمانی از اعداد ۱ تا ۹ در جدول یک حالت از مسئله خواهد بود.
	\item عملیات‌ها:
	کنش‌های مسئله همان جابه‌جا کردن عدد ۹ به یکی از خانه‌های اطرافش خواهد بود. هزینه هر کنش ۱ واحد است اگر هزینه را به صورت تعداد جابه‌جایی‌ها در نظر بگیریم.
	\item شرط رسیدن به هدف:
	چیدمانی از اعداد ۱ تا ۹ که جمع عناصر قرار گرفته روی سطر، ستون و قطر با یکدیگر برابر شود. هر چیدمانی که این شرط را ارضا کند هدف محسوب می‌شود.
\end{itemize}
\section{سوال ۳}
با توجه به این‌که در جست‌و‌جوی گرافی راس‌های دیده شده را دوباره اکسپند نمی‌کنیم، دنباله‌ی راس‌های دیده شده تا رسیدن به یک راس هدف به صورت زیر خواهد بود:
\[
S \to A \to D \to C \to G_1
\]	
	
	
\end{document}