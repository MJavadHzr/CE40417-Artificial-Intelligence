\documentclass[a4paper, 12pt]{article}
\usepackage{temp}
\usepackage{epsfig,graphicx,subfigure,amsthm,amsmath, float, xcolor, changepage, mathtools, textcomp, hyperref, bm, amssymb, tcolorbox, tikz, setspace}
\usepackage{array}
\usepackage[shortlabels]{enumitem}
\usepackage[stable]{footmisc}
\usepackage{xepersian}
\settextfont[Scale=1]{XBZar}
%\setdigitfont{XBZar}
\setlatintextfont[Scale=0.9]{Times New Roman}
\hypersetup{
	colorlinks=true,
	urlcolor=blue!70!black
}

\newcolumntype{?}{!{\vrule width 1pt}}

\doublespacing
\begin{document}
\handout
{هوش مصنوعی}
{نیم‌سال اول ۰۱\lr{-}۰۰}
{دکتر محمدحسین رهبان}
{دانشکده مهندسی کامپیوتر}
{تمرین پنجم - بخش اول}
{محمدجواد هزاره}
{98101074}
\noindent
\\[-6em]
\section*{سوال ۱}
برای تعریف \lr{HMM} نیاز به توزیع احتمال حالات اولیه،
\lr{Transitions}
و
\lr{Emissions}
داریم. با توجه به این‌که هر حالت $k$ مقدار مختلف می‌تواند بگیرد، با مشخص کردن $k-1$ احتمال برای مقادیر می‌توان توزیع احتمال اولیه را شناخت. برای مشخص شدن \lr{Transitions} باید 
$\prob\{S_t\,|\,S_{t-1}\}$
را مشخص کنیم که برای یک مقدار مشخص برای $S_{t-1}$، باید $k-1$ احتمال برای مقادیر $S_t$ را مشخص کنیم؛ بنابراین در مجموع برای مشخص کردن \lr{Transitions} به
$k(k-1)$
پارامتر نیاز داریم. برای مشخص کردن \lr{Emissions} نیز باید 
$\prob\{O\,|\,S\}$
را مشخص کنیم که برای یک مقدار مشخص از $S$، به $m-1$ احتمال برای مقادیر مختلف $O$ نیاز داریم. بنابراین در مجموع برای \lr{Emissions} نیز 
$k(m-1)$
پارامتر نیاز داریم. پس در کل به
$\boxed{k^2 + km - k + 1}$
پارامتر نیاز خواهیم داشت.
\begin{enumerate}[A)]
	\item
	برای محاسبه‌ی احتمال خواسته شده به صورت زیر عمل می‌کنیم:
	\[
	\prob({\scriptstyle O_1=0, O_2=1, O_3=0}) = \prob({\scriptstyle S_3=A, O_1=0, O_2=1, O_3=0}) + \prob({\scriptstyle S_3=B, O_1=0, O_2=1, O_3=0}) \quad (\ast)
	\]
	برای حساب کردن احتمال‌های جدید نیز می‌توان از روش \lr{forward} استفاده کرد. داریم:
	\[
	\prob({\scriptstyle S_3=A, O_1=0, O_2=1, O_3=0}) = \prob({\scriptstyle O_3=0\,|\,S_3=A})\sum_{s}\prob({\scriptstyle S_3=A\,|\,S_2=s})\prob({\scriptstyle S_2=s, O_1=0, O_2=1}) \quad (\star)
	\]
	برای محاسبه‌ی عبارت داخل پرانتز نیز خواهیم داشت:
	\[
	\begin{dcases}
		\prob({\scriptstyle S_2=A, O_1=0, O_2=1}) &= \prob({\scriptstyle O_2=1\,|\,S_2=A})\sum_s\prob({\scriptstyle S_2=A\,|\,S_1=s})\prob({\scriptstyle S_1=s, O_1=0}) \\&= 0.2\big(0.99\times0.8\times0.99 + 0.01\times0.1\times0.01\big) \approx 0.157\\
		\prob({\scriptstyle S_2=B, O_1=0, O_2=1}) &= \prob({\scriptstyle O_2=1\,|\,S_2=B})\sum_s\prob({\scriptstyle S_2=B\,|\,S_1=s})\prob({\scriptstyle S_1=s, O_1=0}) \\&= 0.9\big(0.01\times0.8\times0.99+0.99\times0.1\times0.01\big) \approx  0.008\\
	\end{dcases}
	\]
	بنابراین برای $(\star)$ خواهیم داشت:
	\[
	\begin{aligned}
		\prob({\scriptstyle S_3=A, O_1=0, O_2=1, O_3=0}) &= \prob({\scriptstyle O_3=0\,|\,S_3=A})\sum_{s}\prob({\scriptstyle S_3=A\,|\,S_2=s})\prob({\scriptstyle S_2=s, O_1=0, O_2=1}) \\
		&= 0.8\big(0.99\times0.157 + 0.01\times 0.008\big) \\
		&\approx 0.124
	\end{aligned}
	\]
	به طور مشابه داریم:
	\[
	\begin{aligned}
		\prob({\scriptstyle S_3=B, O_1=0, O_2=1, O_3=0}) &= \prob({\scriptstyle O_3=0\,|\,S_3=B})\sum_{s}\prob({\scriptstyle S_3=B\,|\,S_2=s})\prob({\scriptstyle S_2=s, O_1=0, O_2=1}) \\
		&=0.1\big(0.1\times0.157 + 0.99\times0.008\big) \\
		&\approx 0.002
	\end{aligned}
	\]
	بنابراین برای $(\ast)$ خواهیم داشت:
	\[
	\prob({\scriptstyle O_1=0, O_2=1, O_3=0}) \approx 0.126\quad :)
	\]
	\item
	
	\item
\end{enumerate}
\end{document}



